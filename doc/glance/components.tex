%
% $Id$
%
% Copyright (c)  2007  Tim Niemueller, AllemaniACs RoboCup Team.
%
% Created: Wed Jan 24 2007 15:30:54

\section{Components}

\frame{\tableofcontents[currentsection,hideothersubsections]}

\subsection{At a Glance}
\begin{frame}
  \frametitle{Basic Components}
  \begin{block}{Libraries}
    Basic set of libraries that can be used in Fawkes.
  \end{block}
  \begin{block}{Main Application}
    Application able to load, init and run Fawkes plugins.
  \end{block}
  \begin{block}{Threads}
    Everything is a thread. All operations are carried out in a thread.
  \end{block}
  \begin{block}{Plugins}
    Units containing the real functionality of the robot.
  \end{block}
\end{frame}

\subsection{Libraries}
\begin{frame}
  \frametitle{Library Overview}
  \pgfdeclareimage[height=5cm]{libraries}{figures/libs}
  \begin{center}
    \pgfuseimage{libraries}
  \end{center}
\end{frame}

\begin{frame}
  \frametitle{Fawkes Core Library (FCL, core)}
  \begin{itemize}
  \item Contains core components, may not depend on other non-system libraries
  \item Threading API
  \item Synchronization constructs (mutex, wait condition, read-write lock etc.)
  \item Exception API
  \item Plugin API
  \item Base utils needed in FCL
  \end{itemize}
\end{frame}

\begin{frame}
  \frametitle{Fawkes Utilities (utils)}
  \begin{itemize}
  \item Resembles old utilities
  \item May not depend on other non-system libs besides core
  \item Currently:
    \begin{itemize}
    \item Logging
    \item IPC
    \item Plugin loading
    \item System
    \item Text
    \item Time
    \end{itemize}
  \end{itemize}
\end{frame}

\begin{frame}
  \frametitle{Network Communication Library (netcomm)}
  \begin{itemize}
  \item Socket API
  \item Fawkes Network Protocol implementation
  \item Multicast WorldInfo Transceiver
  \item Robot and service discovery using DNS-based service discovery over
    multicast DNS (mDNS-SD)
  \end{itemize}
\end{frame}

%\begin{frame}
%  \frametitle{Fawkes Configuration Library (config)}
%  \begin{itemize}
%  \item Configuration API
%  \item Implementation using SQLite
%  \end{itemize}
%\end{frame}

\subsection{Infrastructure Components}

\begin{frame}
  \frametitle{BlackBoard (BB, blackboard)}
  \begin{itemize}
  \item Unified and central information storage
  \item Completely new implementation with similar goals
  \item Read/write lock per interface
  \item No more searching, simple pointers, small data blocks (few bytes)
  \item No multi-process access
  \item Central logging instance integrated (tbd)
  \item Possibility to get notified of changes (tbd)
  \item Guarantees writer singleton
  \end{itemize}
\end{frame}

\begin{frame}
  \frametitle{BlackBoard - Interfaces}
  \begin{itemize}
  \item Access to BB data via so-called interfaces
  \item C++ wrapper class with \texttt{read()} and \texttt{write()} operations
  \item Interfaces are instantiated by the InterfaceManager
  \item There may be only one writer instance at any one time per Interface
  \item Protected with Read/Write locks
  \item Interface generator to transform XML descriptions into code
  \end{itemize}
\end{frame}

\begin{frame}
  \frametitle{Configuration Subsystem}
  \begin{itemize}
  \item C++ interface defines access independently of implementation
  \item Currently SQLite implementation exists
  \item Configurations may be tagged, used for instance for different locations
    and backups
  \item Handlers can be registered that are immediately notified of any
    configuration modification
  \item Network protocol and tool implemented to modify configuration
  \item Default and host configuration (overlay to default)
  \end{itemize}
\end{frame}

\begin{frame}
  \frametitle{Network Communication}
  \begin{itemize}
  \item Fawkes Network Protocol implemented
  \item TCP connection, announced and found via mDNS-SD
  \item Arbitrary communication can happen over the Fawkes connection
  \item Currently: PluginManager, ConfigurationManager
  \item BlackBoard communication inside Fawkes (tbd)
  \item Multicast inter-robot communication about world information (wip)
  \end{itemize}
\end{frame}


\subsection{Main Application and Plugins}
\begin{frame}
  \frametitle{Main Application}
  \begin{itemize}
  \item Implements the infrastructure using previously shown components
  \item Uses managers to handle configuration, plugins etc.
  \item Network communication handled by managers or plugins
  \item Fawkes main thread handles timing and basic synchronization
  \item Is the only application run on the robot
  \item Plugins are loaded dynamically
  \end{itemize}
\end{frame}

\begin{frame}
  \frametitle{Plugins}
  \begin{columns}
    \begin{column}{6cm}
      \begin{itemize}
      \item Module for a specific task
      \item Consists of one or more threads
      \item Threads are initialized
      \item It is guaranteed that either all threads are successfully initialized
        or none is started at all
      \end{itemize}
    \end{column}
    \begin{column}{4cm}
      \pgfdeclareimage[width=3cm]{plugin}{figures/plugin}
      \begin{center}
        \pgfuseimage{plugin}
      \end{center}
    \end{column}
  \end{columns}
\end{frame}

\subsection{Aspects}
\begin{frame}
  \frametitle{Aspects and the Tulip Principle}
  \begin{itemize}
  \item<1-> Aspects are wrapped to a thread like the leaves of a tulip's flower
  \item<2-> An Aspect adds a specific functionality to a thread
  \item<2-> A thread may have any number of aspects
  \end{itemize}
  \begin{itemize}
  \item<3-> \emph{BlockedTimingAspect:} thread integrates into the main loop
  \item<3-> \emph{BlackBoardAspect:} thread needs access to the BlackBoard
  \item<3-> \emph{ConfigurableAspect:} thread uses a configuration
  \item<3-> \emph{LoggingAspect:} thread writes output to a logger
  \end{itemize}
\end{frame}

\begin{frame}
  \frametitle{Aspect Initialisation}
  \begin{itemize}
  \item Aspects are initialised by the AspectInitializer
  \item If any error occurs during the initialisation of an aspect the thread
    is never started
  \item Aspects are a clean way to add functionality with minimum overhead
  \item No further initialisation inside the thread needed, just deriving
    the aspect base class is sufficient
  \item Knowledge on how to handle a thread can be derived from the aspects
  \end{itemize}
\end{frame}

\begin{frame}
  \frametitle{Fawkes Diagram}
  \pgfdeclareimage[height=5cm]{fawkes-diagram-1}{figures/fawkes-diagram-1-mainapp}
  \pgfdeclareimage[height=5cm]{fawkes-diagram-2}{figures/fawkes-diagram-2-mainthread}
  \pgfdeclareimage[height=5cm]{fawkes-diagram-3}{figures/fawkes-diagram-3-basic}
  \pgfdeclareimage[height=5cm]{fawkes-diagram-4}{figures/fawkes-diagram-4-plugin}
  \pgfdeclareimage[height=5cm]{fawkes-diagram-5}{figures/fawkes-diagram-5-thread}
  \pgfdeclareimage[height=5cm]{fawkes-diagram-6}{figures/fawkes-diagram-6-aspects}
  \pgfdeclareimage[height=5cm]{fawkes-diagram-7}{figures/fawkes-diagram-7-initthread}
  \pgfdeclareimage[height=5cm]{fawkes-diagram-8}{figures/fawkes-diagram-8-threads}
  \pgfdeclareimage[height=5cm]{fawkes-diagram-9}{figures/fawkes-diagram-9-plugins}
  \begin{center}
    \only<1>{\pgfuseimage{fawkes-diagram-1}}%
    \only<2>{\pgfuseimage{fawkes-diagram-2}}%
    \only<3>{\pgfuseimage{fawkes-diagram-3}}%
    \only<4>{\pgfuseimage{fawkes-diagram-4}}%
    \only<5>{\pgfuseimage{fawkes-diagram-5}}%
    \only<6>{\pgfuseimage{fawkes-diagram-6}}%
    \only<7>{\pgfuseimage{fawkes-diagram-7}}%
    \only<8>{\pgfuseimage{fawkes-diagram-8}}%
    \only<9>{\pgfuseimage{fawkes-diagram-9}}%
  \end{center}
\end{frame}

%%% Local Variables: 
%%% mode: latex
%%% TeX-master: "fawkes-glance"
%%% End: 
